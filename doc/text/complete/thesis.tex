%%%%%%%%%%%%%%%%%%%%%%%%%%%%%%%%%%%%%%%%%%%%%%%%%%%%%%%%%%%%%%%%%%%%%%%%%%%%%
%%% LaTeX-Rahmen fuer das Erstellen von englischen Bachelorarbeiten
%%%%%%%%%%%%%%%%%%%%%%%%%%%%%%%%%%%%%%%%%%%%%%%%%%%%%%%%%%%%%%%%%%%%%%%%%%%%%

%%%%%%%%%%%%%%%%%%%%%%%%%%%%%%%%%%%%%%%%%%%%%%%%%%%%%%%%%%%%%%%%%%%%%%%%%%%%%
%%% allgemeine Einstellungen
%%%%%%%%%%%%%%%%%%%%%%%%%%%%%%%%%%%%%%%%%%%%%%%%%%%%%%%%%%%%%%%%%%%%%%%%%%%%%

\documentclass[twoside,12pt,a4paper]{report}
%\usepackage{reportpage}
\usepackage[english]{babel}
\usepackage{epsf}
\usepackage{graphics, graphicx}
\usepackage{latexsym}
\usepackage[margin=10pt,font=small,labelfont=bf]{caption}
\usepackage[utf8]{inputenc}
\usepackage{amsmath}
\usepackage{amsfonts}
\usepackage{amssymb}
\usepackage{amstext}
\usepackage[numbers]{natbib}
\usepackage{titling}
\usepackage{hyperref}
\usepackage{caption}
\usepackage{subcaption}
\usepackage{multirow}

\textwidth 14cm
\textheight 22cm
\topmargin 0.0cm
\evensidemargin 1cm
\oddsidemargin 1cm
%\footskip 2cm
\parskip0.5explus0.1exminus0.1ex

% Kann von Student auch nach pers\"onlichem Geschmack ver\"andert werden.
\pagestyle{headings}

\sloppy

\begin{document}

%%%%%%%%%%%%%%%%%%%%%%%%%%%%%%%%%%%%%%%%%%%%%%%%%%%%%%%%%%%%%%%%%%%%%%%%%%%%
%%% Layout Title page
%%%%%%%%%%%%%%%%%%%%%%%%%%%%%%%%%%%%%%%%%%%%%%%%%%%%%%%%%%%%%%%%%%%%%%%%%%%%
 
\begin{titlepage}
 \begin{center}
  {\LARGE Eberhard Karls Universit\"at T\"ubingen}\\
  {\large Mathematisch-Naturwissenschaftliche Fakultät \\
Wilhelm-Schickard-Institut f\"ur Informatik\\[4cm]}
  {\huge Bachelor Thesis Bioinformatics\\[2cm]}
  {\Large\bf  Semi-supervised Learning for Nucleic Acid Cross-linking Mass Spectrometry\\[1.5cm]}
 {\large Emil Paulitz}\\[0.5cm]
14.08.2020\\[3cm]
\begin{center}
{\small\bf Reviewer}\\[0.5cm]
 {\large Prof. Oliver Kohlbacher}\\
  {\footnotesize Department of Computer Science\\
	University of T\"ubingen}
  \end{center}
	
\begin{center}
{\small\bf Supervisor}\\[0.5cm]
  {\large Timo Sachsenberg}\\
  {\footnotesize Applied Bioinformatics Group\\
	University of T\"ubingen}\end{center}

  \end{center}
\end{titlepage}
%%%%%%%%%%%%%%%%%%%%%%%%%%%%%%%%%%%%%%%%%%%%%%%%%%%%%%%%%%%%%%%%%%%%%%%%%%%%
%%% Layout back of title page
%%%%%%%%%%%%%%%%%%%%%%%%%%%%%%%%%%%%%%%%%%%%%%%%%%%%%%%%%%%%%%%%%%%%%%%%%%%%

\thispagestyle{empty}
\vspace*{\fill}
\begin{minipage}{11.2cm}
\textbf{Paulitz, Emil:}\\
\emph{Semi-supervised learning for nucleic acid cross-linking mass spectrometry}\\ Bachelor Thesis Bioinformatics\\
Eberhard Karls Universit\"at T\"ubingen\\
Period: 14.04.2020-14.08.2020
\end{minipage}
\newpage

%%%%%%%%%%%%%%%%%%%%%%%%%%%%%%%%%%%%%%%%%%%%%%%%%%%%%%%%%%%%%%%%%%%%%%%%%%%%

\pagenumbering{roman}
\setcounter{page}{1}

%%%%%%%%%%%%%%%%%%%%%%%%%%%%%%%%%%%%%%%%%%%%%%%%%%%%%%%%%%%%%%%%%%%%%%%%%%%%
%%% Page I: Abstract
%%%%%%%%%%%%%%%%%%%%%%%%%%%%%%%%%%%%%%%%%%%%%%%%%%%%%%%%%%%%%%%%%%%%%%%%%%%%


\section*{Abstract}

Modern analysis instruments in life sciences produce data on an ever-larger scale. While this allows more comprehensive insight into biological mechanisms, it also requires computational methods able to process the data appropriately. Currently, the method of choice for analyzing a proteome, the entirety of proteins in a cell or an organism, is mass spectrometry, which produces spectra containing information corresponding to the peptide's amino acid sequence. An important step in matching every spectrum to the peptide they originate from, is the well-established Percolator algorithm. It combines every information about the spectrum, peptide and scores for the similarity of peptide-spectrum-match, to achieve best possible separation between correct and incorrect matches.\\
In this thesis, the algorithm was transferred from C++ into the python programming language. This implementation increases the flexibility and extends the range of potential applications by enabling the use of machine learning libraries like scikit-learn, which can be utilized to adapt the algorithm more easily to the special demands of one’s experiment. As such, we present Pycolator, an adaptation of Percolator to the characteristics of peptides cross-linked onto (ribo-)~nucleotides, for which the original algorithm often does not provide optimal results. Pycolator finds $1\%$ more cross-linked peptides on the tested dataset and employs techniques to reduce variation. The analysis of such proteins promises to yield insight about the interaction between peptides and DNA or RNA, possibly providing a more comprehensive view on the interactions driving biological systems.

\newpage
%%%%%%%%%%%%%%%%%%%%%%%%%%%%%%%%%%%%%%%%%%%%%%%%%%%%%%%%%%%%%%%%%%%%%%%%%%%%
%%% Page 2: Danksagung
%%%%%%%%%%%%%%%%%%%%%%%%%%%%%%%%%%%%%%%%%%%%%%%%%%%%%%%%%%%%%%%%%%%%%%%%%%%%
\section*{Acknowledgements}
I would like to express my gratitude to my supervisor Timo Sachsenberg, who over all of our online meetings had input, ideas and inspiration, never tiring of explaining unknown concepts and even responded to my questions on weekends.\\
%I would also like to thank Leo.org~\footnote{https://dict.leo.org/englisch-deutsch/} for always finding the right words for me.
I also want to thank my proof-readers Jonathan Paulitz (especially for all the personal and general tips), Benedikt Rössler and Simon Schäfermann, as well as Carmen Gil Bredehöft for the best colors, fonts and the perfect refreshment in critical moments.\\
Lastly, I want to thank my parents for their continuous support and love, and for enabling me the studies I cherish. Also my cats for keeping me company and warming me every 10~minutes in the final phase of writing.\\
\vspace*{\fill}\\
In accordance with the standard scientific protocol, I will use the personal pronoun
\textit{we} to indicate the reader and the writer or my scientific collaborators and myself.
\cleardoublepage

%%%%%%%%%%%%%%%%%%%%%%%%%%%%%%%%%%%%%%%%%%%%%%%%%%%%%%%%%%%%%%%%%%%%%%%%%%%%%
%%% Table of Contents
%%%%%%%%%%%%%%%%%%%%%%%%%%%%%%%%%%%%%%%%%%%%%%%%%%%%%%%%%%%%%%%%%%%%%%%%%%%%%

\renewcommand{\baselinestretch}{1.3}
\small\normalsize

\tableofcontents

\renewcommand{\baselinestretch}{1}
\small\normalsize

\cleardoublepage

%%%%%%%%%%%%%%%%%%%%%%%%%%%%%%%%%%%%%%%%%%%%%%%%%%%%%%%%%%%%%%%%%%%%%%%%%%%%%
%%% List of Figures
%%%%%%%%%%%%%%%%%%%%%%%%%%%%%%%%%%%%%%%%%%%%%%%%%%%%%%%%%%%%%%%%%%%%%%%%%%%%%

\renewcommand{\baselinestretch}{1.3}
\small\normalsize

\addcontentsline{toc}{chapter}{List of Figures}
\listoffigures

\renewcommand{\baselinestretch}{1}
\small\normalsize

\cleardoublepage

%%%%%%%%%%%%%%%%%%%%%%%%%%%%%%%%%%%%%%%%%%%%%%%%%%%%%%%%%%%%%%%%%%%%%%%%%%%%%
%%% List of tables
%%%%%%%%%%%%%%%%%%%%%%%%%%%%%%%%%%%%%%%%%%%%%%%%%%%%%%%%%%%%%%%%%%%%%%%%%%%%%

\renewcommand{\baselinestretch}{1.3}
\small\normalsize

\addcontentsline{toc}{chapter}{List of Tables}
\listoftables

\renewcommand{\baselinestretch}{1}
\small\normalsize

\cleardoublepage

%%%%%%%%%%%%%%%%%%%%%%%%%%%%%%%%%%%%%%%%%%%%%%%%%%%%%%%%%%%%%%%%%%%%%%%%%%%%%
%%% List of abbreviations
%%%%%%%%%%%%%%%%%%%%%%%%%%%%%%%%%%%%%%%%%%%%%%%%%%%%%%%%%%%%%%%%%%%%%%%%%%%%%

\addcontentsline{toc}{chapter}{List of Abbreviations}
\chapter*{List of Abbreviations\markboth{LIST OF ABBREVIATIONS}{LIST OF ABBREVIATIONS}}

\begin{tabbing}
\textbf{FACTOTUM}\hspace{1cm}\=Schrott\kill
\textbf{MS}\> Mass Spectrometry \\
\textbf{LC} \> Liquid Chromatography\\
\textbf{MS/MS} \> Tandem Mass Spectrometry\\
\textbf{PSM} \> Peptide Spectrum Match\\
\textbf{FDR} \> False Discovery Rate\\
\textbf{ROC} \> Receiver Operating Characteristics\\
\textbf{TPR} \> True Positive Rate\\
\textbf{FPR} \> False Positive Rate\\
\textbf{SVM} \> Support Vector Machine\\
\textbf{AUC} \> Area Under the Curve\\
\end{tabbing}

\cleardoublepage

%%%%%%%%%%%%%%%%%%%%%%%%%%%%%%%%%%%%%%%%%%%%%%%%%%%%%%%%%%%%%%%%%%%%%%%%%%%%%
%%% Der Haupttext, ab hier mit arabischer Numerierung
%%% Mit \input{dateiname} werden die Dateien `dateiname' eingebunden
%%%%%%%%%%%%%%%%%%%%%%%%%%%%%%%%%%%%%%%%%%%%%%%%%%%%%%%%%%%%%%%%%%%%%%%%%%%%%

\pagenumbering{arabic}
\setcounter{page}{1}

%% Introduction and Background
%%%%%%%%%%%%%%%%%%%%%%%%%%%%%%%%%%%%%%%%%%%%%%%%%%%%%%%%%%%%%%%%%%%%
% Einleitung
%%%%%%%%%%%%%%%%%%%%%%%%%%%%%%%%%%%%%%%%%%%%%%%%%%%%%%%%%%%%%%%%%%%%

\chapter{Introduction}
\label{introduction}
- Motivation
%Am Ende der Einleitung folgt ein Text so \"ahnlich wie (Nat\"urlich das folgende alles auf Englisch !!):
%
%Die Arbeit gliedert sich dazu wie folgt: Die Grundlagen von ...
%werden in Kapitel~\ref{introduction} erarbeitet. 
%...
%Eine
%Diskussion und ein kurzer Ausblick im
%Kapitel~\ref{discussion} beschlie{\ss}en diese Arbeit.
%
%Bevor wir uns der Auswertung bzw. Bewertung der gewonnenen Prim\"ardaten zuwenden, wollen wir zun\"achst einige grundlegende Begriffe der deskriptiven Statistik wiederholen.

\section{Background}
\label{background}

	Proteomics is an interdisciplinary research field analyzing the composition, interaction and impacts of the proteome (the entirety of proteins) of single cells or up to a whole organism~\cite{Han2008, Sachsenberg2017}. In this thesis, research was done in a related field, focusing on peptides cross-linked with RNA. The chemical bond between cross-linked molecules has been artificially induced, for example using UV light~\cite{Sachsenberg2017}. Applying this to peptides and RNA could possibly give insight into their \textit{in vivo} interactions, and may also allow conclusions about protein-DNA interaction.\\
	For quantitatively characterizing the proteome of a sample, large scale measuring techniques are needed. Mostly, mass spectrometry (MS) is used, or more specifically, as for the data in this thesis, tandem mass spectrometry (MS/MS) combined with liquid chromatography (LC). In order to analyze the protein sample with MS, its complexity has to be reduced as much as possible, for example using LC~\cite{Sachsenberg2017}. As~\citet{Han2008} explains, the mass spectrometer then produces mass spectra, which have to be analyzed further. It does so by first ionizing the substrate, because it can only detect charged particles. Then, the sample is separated in the mass analyzer by the ratio $\frac{m}{z}$, mass of the particles to their charge. The detector then quantifies the amount of a particle in the sample. The result is a mass spectrum, as shown in figure~\ref{fig:mass_spectrum}.\\
	{
	\renewcommand{\baselinestretch}{0.9} 
	\normalsize
	\begin{figure}
		\centering
		\includegraphics[width = \textwidth, trim=3.9cm 20cm 2.5cm 3cm,clip]{figures/Grafik_Timo.pdf}
		\label{fig:mass_spectrum}
		\caption[Example for a mass spectrum]{Example for a mass spectrum as recorded by a mass spectrometer. The ion intensity correlates with the amount of a molecule in the sample, $\frac{m}{z}$ is the mass-to-charge-ratio. From:~\citet{Sachsenberg2017}}
	\end{figure}
	}
	Because mass alone does not give enough information about a peptide to determine its sequence, tandem mass spectrometry is often used to gather more detailed evidence. In this procedure, particles of similar $\frac{m}{z}$ ratio are selected for fragmentation in a collision cell after the first round of mass measurement~\cite{Sachsenberg2017}. In there, the substance collides with a gas to be broken down into smaller molecules. For proteins, fragmentation happens predominantly in their backbone, producing all possible sub-sequences of the peptide. This produces a spectrum that is almost unique for its protein, which allows for peptide identification using bioinformatics tools~\cite{Angel2012}.\\
	Algorithms like Sequest~\cite{Eng1994} or X!~Tandem~\cite{Craig2004} compare the resulting spectra with theoretical spectra calculated from a list of possible peptides and compute a score based on their similarity. The peptides are generated by obtaining a list of proteins expected in the sample and calculating the peptides resulting from the, for example enzyme-based, degradation of the proteins. The best scoring peptide is then considered a peptide-spectrum-match (PSM). The scores produced by those algorithms often do not distinguish well enough between correct and incorrect matches~\cite{Kll2007}, but they enable FDR estimation using decoy databases and serve as a basis for score re-calibration with the Percolator~algorithm~\cite{Kll2007, Granholm2012}.\\
	Decoy databases are created from the target database, contain usually as  many peptides~\cite{Peng2003, Moore2002} in a reversed or shuffled order with respect to the amino acid sequence~\cite{Aggarwal2016}. They are presented to the scoring algorithm either separately~\cite{Granholm2012} or mixed with the target database~\cite{Peng2003}. It is assumed, that decoy and target peptides have similar features~\cite{Moore2002} and are not easily distinguishable by a scoring algorithm. When the actually fitting peptide for a given spectrum is not in the target database, and thus a wrong one will be chosen, the best scoring peptide will be a decoy approximately half of the time. This allows for an estimation of wrongly assigned targets, since the score distribution is assumed to be the same for decoys and false targets~\cite{Aggarwal2016}.\\
	In practice, one estimates the probability of a PSM being a false target by counting the number of decoy-PSMs with the same or a higher score. It is then assumed, there are as many false targets and thus a false discovery rate (FDR) can be estimated. This leads to the following formula\footnote{In this thesis, the following approximation is used:\\
		$FDR \approx \frac{\text{\# decoy PSMs}}{\text{\# all PSMs}} = \frac{\text{\# decoy PSMs}}{\text{\# decoy PSMs} + \text{\# target PSMs}}$\\
		It is faster to calculate and yields results differing by the FDR, so in the relevant range of FDRs of $0$ to $5\%$ up to $5\%$:\\
		$\frac{\frac{\text{\# decoys}}{\text{\# targets}}}{\frac{\text{\# decoys}}{\text{\# decoys} + \text{\# targets}}} = \frac{\text{\# decoys} + \text{\# targets}}{\text{\# targets}} = 1 + \frac{\text{\# decoys}}{\text{\# targets}} \approx 1 + FDR$}~\cite{Granholm2012}:
	\begin{equation}
	FDR = \frac{\text{\# false target PSMs}}{\text{\# all target PSMs}} \approx  \frac{\text{\# decoy PSMs}}{\text{\# all target PSMs}}
	\end{equation}
	The q-value as a measure for a single PSM rather than a metric for a set of PSMs is then derived from this as the minimum FDR of all PSMs with a lower or equal score~\cite{Granholm2012, Aggarwal2016}. It will be used for estimating the credibility for any one PSM.\\
	As \citet{Kll2007} say, separating correct from incorrect target PSMs with already mentioned algorithms works fine, but there is still room for improvement. This is because often not all information is used and considered jointly. Percolator~\cite{Kll2007, Granholm2012} tries to utilize as much information as possible by using scores from different algorithms, features of the peptide like its length, of the spectrum or the PSM itself. It joins them using a linear SVM and a semi-supervised approach with cross-validation to retain as many PSMs as possible. In every iteration, the top ranking, non-decoy PSMs up to a certain threshold of q-value are chosen as positive training examples, and the decoy PSMs are used as negative training set. The PSMs are then re-ranked using the SVM score, with the intend of getting a better separation of true and false PSMs. If that holds true, the positive training set of the next iteration better is of higher quality and the SVM can be trained even better. The algorithm usually converges within the first 10 iterations~\cite{Kll2007}. To avoid having to split the data into training and testing set and consequently losing possibly correct PSMs but also avoid overfitting, a nested cross-validation approach is being used~\cite{Granholm2012}.
	- ROC Curve noch erklären?

\cleardoublepage

%% Material and Methods
%%%%%%%%%%%%%%%%%%%%%%%%%%%%%%%%%%%%%%%%%%%%%%%%%%%%%%%%%%%%%%%%%%%%
% Grundlagen
%%%%%%%%%%%%%%%%%%%%%%%%%%%%%%%%%%%%%%%%%%%%%%%%%%%%%%%%%%%%%%%%%%%%

\chapter{Material and Methods}
\label{matmet}

- Material: Was ich für ein Datensatz zum Testen benutzt habe und wo der herkommt

\section{Implementation of the percolator algorithm}
- Wie genau stelle ich das vor, siehe Mail? Wichtige Punkte wären:\\
- Verwendete Scipy-Methoden\\
- Abbruch wenn es nicht besser wird und dass ich die AUC als Metrik nutze\\
- feature normalization\\
- Wichtige Hilfsfunktionen (pseudoROC zB)

\section{Improvements of the percolator algorithm for cross-link identification}
- Klassenspezifische q-values\\
- ROC nach jeder Iteration und aufgesplittet nach XL/nXL\\
\subsection{How to deal with different Ranks}
- OptimalRanking Option (Erst paar Iterationen Ränge verändern lassen und dann die schlechten entfernen)\\
\subsection{Characteristics of cross-linking PSM datasets}
- Verhältnis Targets:Decoys und XL:non-XL in inneren und äußeren splits gleich lassen und MinMaxMedian Auswertungen mithilfe von google colab cloud computing\\
- Imputation (kam zwar nichts raus ist aber trotzdem interessant)\\
- Trennung von Datensatz nach XL/nXL oder sogar cross-linking target falls Datensatz groß genug\\
\subsection{Small datasets}
- Ratio Testing (nicht-random aus ganzem Datensatz und random aus Top 10\%. Liefert Erkenntnisse über die mögliche Größe des Datensatzes und eventuell die Sinnhaftigkeit, wann man die Datensätze einfach trennen kann $\rightarrow$ Für den Leser relevant)\\
- Einbau von Identifikationen bei 1\% FDR als Metrik (Sinnhaftigkeit kann man ja diskutieren)\\\\			
(- Performance auf anderem Datensatz\\
- Vergleich mit Entrapment FDR)

%{
%\renewcommand{\baselinestretch}{0.9} 
%\normalsize
%\begin{table}[htb]
%\begin{tabular}{|c|}
%\end{tabular}
%  \caption[Tabellenverzeichnis]{}
%  \label{tab:1}
%\end{table}
%}
\cleardoublepage

%% Results
\chapter{Results}
\label{results}


\section{Implementation of the percolator algorithm}
- Reimplementierung funktioniert wie Original\\
- feature normalization war wichtiger boost\\
- ROC nach jeder Iteration zeigen

\section{Improvements of the percolator algorithm for cross-link identification}
\subsection{How to deal with different Ranks}
- Ergebnisse von OptimalRanking

\subsection{Characteristics of cross-linking PSM datasets}
- Verhältnis Targets:Decoys und XL:non-XL verringt die Streuung: MinMaxMedian Auswertungen\\
- Bei Imputation kam nichts heraus\\
- Großer Unterschied wenn man den (großen) Datensatz nach XL/nXL oder sogar cross-linking target aufteilt\\

\subsection{Small datasets}
- Sinnvolle Plots zu Ratio Testing\\
- Neue Metrik erlaubt es der Implementierung, auch auf kleineren Datensätzen zu funktionieren
\cleardoublepage

%% Discussion and Outlook
%%%%%%%%%%%%%%%%%%%%%%%%%%%%%%%%%%%%%%%%%%%%%%%%%%%%%%%%%%%%%%%%%%%%
% Diskussion und Ausblick
%%%%%%%%%%%%%%%%%%%%%%%%%%%%%%%%%%%%%%%%%%%%%%%%%%%%%%%%%%%%%%%%%%%%

\chapter{Discussion}
\label{discussion}
As already explained in~\ref{lab:matmet:ranks}, cross-linked peptides often are harder to score than linear ones. Therefore, they can get a lower score than appropriate and it can be useful to also include the best scoring cross-linked peptide when running the Percolator algorithm. It may be able to revise the PSMs scores and the actually correct cross-linked peptide may become the best scoring PSM. This thesis is also supported by the findings in~\ref{lab:results:ranks}. If only including the best scoring peptide, of which~\ref{fig:only_rank_one} shows the results, the end result is worse than when giving Pycolator some iterations to re-rank the found PSMs. However, it was better than when giving Pycolator all of the PSMs available, which is unexpected, since giving a machine learning model more information should generally improve its learning. Apparently, the lower ranking PSMs, even when having such a high score a q-value of $\leq5\%$ is estimated, contain misleading information and thus the SVM learns patterns not valid for correct PSMs. This also explains why the algorithm converges faster when only given rank 1 PSMs. The higher quality of data lets the SVM learn the correct patterns after fewer iterations. \\
The pseudo ROC generated when using the newly implemented mechanism~(\ref{fig:optimalranking}) shows a convergence after $5$ iterations, just like when using every PSM~(\ref{fig:all_ranks}). Then, as the log shows, lower ranking PSMs are dropped and the next iteration has a much higher AUC, probably as a result of the better quality of the PSMs. Letting the algorithm run on the new dataset again improves the AUC even beyond that of Pycolator when only using rank 1 PSMs. This suggests, that indeed a re-ranking takes place in the first half of iterations.\\
Comparing the AUC of Pycolator when run with the new mechanism~(\ref{fig:optimalranking}) after iteration $6~(345.26$) with the end result of running Pycolator with every PSM available~(\ref{fig:all_ranks},~$343.21$), yields the following insight: Dropping all PSMs with rank $2+$ yields a worse result when the Percolator algorithm has been running for 10 rather than 6 iterations. This implies an overfitting onto the PSMs with lower quality and thus a worse scoring.\\\\
Although implementing the balancing of different classes in the nested cross-validation splits had no big impact, it reduced the spread. On other datasets, containing much fewer cross-linked than non-cross-linked PSMs, this procedure could have greater impact. This needs to be tested in further experiments. The feature has no disadvantages except negligible worse performance and can be generalized to other classes that could be present in a dataset.\\
Using imputation did not improve the performance of Pycolator. Neither for a single class nor for both. It is thinkable that imputation has a different effect on other datasets, which contain far fewer cross-links than non-cross-links. It could then prevent that the SVM fits onto the characteristics of non-cross-links because they are more frequent and find very few cross-links, but rather fit onto both classes in the same way. This has to be tested in further experiments. Because the imputation takes a significant amount of time, this option is switched off per default.\\
Splitting the dataset by cross-linked and non-cross-linked PSMs slightly improved the performance of Pycolator, and splitting by the cross-linked nucleotide yielded a great improvement. In practice, neither is applicable to most datasets, because they often only contain a very small number of cross-linked PSMs. As the experiments in section~\ref{lab:results:small_datasets} show, applying Pycolator on a small dataset bring a lot of variation.\\
The non-random approach in the first experiment regarding small datasets keeps the ratio of correct to incorrect PSMs and ensures the appearance of PSMs originally classified as correct. However, maintaining the ratio when sampling a very small subset means it will contain one or two PSMs, which originally were assigned a q-value of $<1\%$, and many worse PSMs. This general bad quality of the dataset makes it difficult to obtain quantitative results for the very small datasets, which is why the second experiment was conducted. The findings however do suggest that down to a dataset size of about 500, Pycolator does not suffer from any drawbacks. With smaller datasets, the AUC metric does not work anymore and a lot of variation happens~(\ref{fig:results:small_dataset_first_auc_ratio_pxl}). Only the first problem could be solved: by implementing the new metric of identifications at $1\%$~q-value. Figure~\ref{fig:results:small_dataset_first_ratio_dxl} was generated to test if the FDR estimation would suffer from small datasets. This can not be observed, as the smallest dataset for which a point is plotted has approximately $100$ PSMs, and thus the small increase from $0.061$ to $0.07$ is due to rounding.\\
In the second experiment however, this thesis seems to be supported by figure~\ref{fig:results:small_dataset_snd_found_dxl}, showing that when re-calculating the q-value after sampling, without changing the score, more PSMs receive a high confidence. But this is likely due to the fact that few high-scoring decoys determine the q-value of many high-scoring targets. When one samples randomly, the chance that many of those decoys will not be sampled increases with a smaller sampling size, enhancing the FDR estimation of the targets, which usually lie above $1\%$~q-value.\\
As figure~\ref{fig:results:small_dataset_snd_comparison} shows, with the whole dataset the Pycolator~score achieves a significantly better separation of true and false PSMs than the NuXL~score. However, the smaller the given dataset gets, the smaller this advantage becomes, rarely Pycolator performs even worse.

- Ist Pycolator besser geworden?\\
- Neues Diagramm zeichnen\\
- Methoden hinterfragen oder begründen, Ergebnisse interpretieren, Anwendbarkeit diskutieren, z.B.:\\
- Falsche Formel für q-value\\
- C Parameter für jeden split neu optimieren führt zu overfitting? $\rightarrow$ Original-Algorithmus macht es auch so\\
- Wie sinnvoll ist die neue Metrik (idents bei 1\%)?\\
- ScanNr Versuche: Gleiche Spektren (identifiziert anhand der ScanNr.) auf verschiedene splits verteilen verändert nichts, d.h. vermutlich sind die niedrigeren Ränge dann so schlecht, dass es nichts bringt die schonmal gesehen zu haben.\\
- Peptide Versuche: Schlechtere Ergebnisse, aber vllt ehrlicher?\\
%Die Generalisierung von manchen Peptiden auf alle Pepide fällt der SVM schwer, weil der score sinkt wenn man die Peptide im test set nicht auch im train set zeigt.\\
%auf was soll man aber noch generalisieren, die svm wird gelöscht sobald der split fertig ist. Ist es hier unehrlich oder besseres training, wenn man der svm gute Beispiele zeigt?\\		
%Idee: Vielleicht sind es nur wenige, bestimmte Peptide (oder Proteine), die besondere Eigenschaften haben und somit schlecht vorhergesagt werden können?\\
%Oder: Peptide von decoys und peptide von targets sind disjunkt. Vielleicht haben gleiche Peptide gleiche Eigenschaften in manchen der scores, und die SVM kriegt diese Eigenschaften über die false train mit? Und kann somit zwischen decoy und target direkt unterscheiden?\\
%$\rightarrow$ weitere Experimente? Zu beachten: nXL werden hier immer schlechter!\\
\section{Outlook}
- Mögliche weiterführende Experimente: mächtigere Klassifikatoren + monotonic constraints (wie von Timo ausprobiert), Ada-Boosting, feature selection

\cleardoublepage


%%%%%%%%%%%%%%%%%%%%%%%%%%%%%%%%%%%%%%%%%%%%%%%%%%%%%%%%%%%%%%%%%%%%%%%%%%%%%
%%% Bibliographie
%%%%%%%%%%%%%%%%%%%%%%%%%%%%%%%%%%%%%%%%%%%%%%%%%%%%%%%%%%%%%%%%%%%%%%%%%%%%%

\addcontentsline{toc}{chapter}{Bibliography}

\bibliographystyle{plainnat}
\bibliography{bibliography}

\cleardoublepage

%%%%%%%%%%%%%%%%%%%%%%%%%%%%%%%%%%%%%%%%%%%%%%%%%%%%%%%%%%%%%%%%%%%%%%%%%%%%%
%%% Erklaerung
%%%%%%%%%%%%%%%%%%%%%%%%%%%%%%%%%%%%%%%%%%%%%%%%%%%%%%%%%%%%%%%%%%%%%%%%%%%%%
\thispagestyle{empty}
\section*{Selbst\"andigkeitserkl\"arung}

Hiermit versichere ich, dass ich die vorliegende Bachelorarbeit 
selbst\"andig und nur mit den angegebenen Hilfsmitteln angefertigt habe und dass alle Stellen, die dem Wortlaut oder dem 
Sinne nach anderen Werken entnommen sind, durch Angaben von Quellen als 
Entlehnung kenntlich gemacht worden sind. 
Diese Bachelorarbeit wurde in gleicher oder \"ahnlicher Form in keinem anderen 
Studiengang als Pr\"ufungsleistung vorgelegt. 

\vskip 3cm

Ort, Datum	\hfill Unterschrift \hfill 


%%% Ende
%%%%%%%%%%%%%%%%%%%%%%%%%%%%%%%%%%%%%%%%%%%%%%%%%%%%%%%%%%%%%%%%%%%%%%%%%%%%%

\end{document}

